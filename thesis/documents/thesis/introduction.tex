
\pagenumbering{arabic} % ab jetzt arabische Nummerierung

\chapter{Introduction}
%\section{Importance and challenges of segmentation in minmal invasive surgery}
%\section{Improving the segmentation through generation of artificial smoke}
Minimally invasive surgery (MIS) is a widely used surgery technique, because it offers several advantages over traditional open surgery \cite{mohiuddin2013maximizing}. 
In MIS, surgeons make small incisions and use specialized tools to access and operate only on the affected area.
This stands in contrast to open surgery, where gaining access is achieved through large incisions and opening up the body cavity.
This way MIS induces less trauma, reduces blood loss, and ensures a faster recovery \cite{mohiuddin2013maximizing}.\\
However, these benefits also come with challenges.
One of which would be that the surgeon`s view of the operating field is often limited and can only be seen through an endoscope.
This endoscope must always point at the surgical procedure, whereby the adjustment of this endoscope has to be often done manually.
Therefore there are approaches to making the endoscope guidance robot-assisted, for which a segmentation of the received camera output is essential \cite{gruijthuijsen2022robotic}.\\
This in turn brings its own challenges, as the quality of the images can be degraded by interfering influences.
These interfering influences include reflections, motion artifacts, and smoke produced by cutting with an electrical cutter.
This makes it difficult to control the endoscope in a robot-assisted way and harder for the surgeons to accurately interpret the images, which results in the need for an accurate segmentation that is less likely to be disturbed through interferences.
Since one of the main interference factors is smoke, an approach would be to improve the segmentation in the presence of it.\\
This thesis proposes a solution to improve the performance of segmentation models in MIS by generating synthetic images with specific characteristics.
Specifically, the generation of artificially smoked images is explored as a means to increase the amount of training data for segmentation models, improving their ability to cope with challenging visual conditions such as smoke interference.
An I2I translation model is trained to transform original images into output images with the desired smoked appearance, which are then used to augment the training data for a state-of-the-art segmentation network, DeepLabv3+ \cite{Chen2018a}. 
Two I2I translation models, CycleGan \cite{Zhu2017} and StarGan \cite{choi2018stargan}, are hereby used and evaluated for generating artificial smoke images. 
A new network, GenSegNet, is proposed to include segmentation in the image generation process. 
GenSegNet is trained to reward the generative network for producing images that are difficult to segment, which in turn are used to further train the segmentation network.
We hypothesize that using the proposed GenSegNet model will show further improvement in segmentation performance.
The results of the experiments are evaluated using six-fold cross-validation.\\
In this thesis firstly a theoretical foundation for machine learning is given for understanding in chapter \ref{theoreticalfound}.
Afterwards, the used artificial neural networks (ANNs) are explained in chapter \ref{method}.
Then the different experiments and their results are presented in chapter \ref{result}.
At last, an interpretation of the results is given in chapter \ref{discussion}, showing how I2I translation to a domain with heavy smoke can help improve segmentation in MIS. 
Here the limitations of the models and metrics are also addressed.
%-> contribution

